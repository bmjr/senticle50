\documentclass[11pt]{report}
\usepackage{standalone}
\graphicspath{ {images/} }
\setcounter{tocdepth}{5}
\setcounter{secnumdepth}{5}
\usepackage{natbib}
\bibliographystyle{agsm}
\usepackage{etoolbox}
\setlength{\parindent}{0em}
\setlength{\parskip}{0.25em}
\usepackage[raggedright]{titlesec}
\usepackage{hyperref}
\usepackage{capt-of}
\patchcmd{\bibliography}{\section*}{\section}{}{}
\titlespacing*{\chapter}{0pt}{-40pt}{10pt}
\titleformat{\chapter}[block]{\normalfont\huge\bfseries}{\thechapter}{15pt}{}
\usepackage{rotating}
\usepackage[titletoc]{appendix}
\usepackage{pgfgantt}
\usepackage{graphicx, rotating, caption, lscape, threeparttable}% \usepackage{amsmath}
\usepackage{array}
\usepackage{pdflscape}
\usepackage{geometry}
\usepackage{listings}
\usepackage[T1]{fontenc}




\begin{document}
\chapter{Analysis}

\section{Requirements Elicitation}
To ensure the complete set of system requirements was gathered two methods of requirement capture were employed which include Observation and Inspection of Existing System Documentation. Each method allowed for the discovery of a smaller subset of requirements that when combined together gave a complete overview of what is required from the system.

\subsection*{Observation}
The existing Brexit visual sentiment system hosted by Edinburgh \citep{llewellyn_brexit?_2016} was observed to discover the core visual requirements of a semantic visualizer and areas of human computer interaction that required improvement in the Edinburgh system were noted as requirements of this system e.g. the granularity of different time levels rather than just having day and overview. 

\subsection*{Inspection of Existing System Documentation}
To gain an understanding of system features and what a classification framework requires, the existing documentation of real time classification system's by Edinburgh \citep{llewellyn_brexit?_2016} and Sheffield \citep{maynard_framework_2017} was analysed.

\section{Specification}
The requirements discovered within the requirements elicitation were formalised. Each of the requirements were then checked to ensure they were clear, unambiguous and verifiable. After which they were structured into the below software requirements specification.

\chapter*{Software Requirements Specification}

\subsection*{Product Functions}
\begin{itemize}
\item The System must be able to collect Tweets on the topic of ``Britain leaving the European Union''.
\item The System must be able to classify the sentiment of tweets that is has received.
\item The System must be able to visually display the sentiment of the public
on the topic of ``Britain leaving the European Union''.
\item The System must collect, classify and visualize the tweets within sufficient real time.
\item The System must be autonomous in it's operations.
\end{itemize}

\subsection*{User Classes and Characteristics}
\textbf{Operator}: An Individual who has a reasonable technical expertise and wants to interface with the System directly via command line tools and code itself.

\textbf{General User}: An individual who wishes to inspect the visual analysis that the system computes. This user is encompasses all the possible characteristics of a miscellaneous member of the general public wishing to engage with the system.

\subsection*{Operating Environment}
The System will operate on a remote virtual private server with an operating system of a Ubuntu Linux Distribution. 

The System will require a data store for permanent and temporary storage of data with the data store being maintainable, scalable and available at all times.

\subsection*{Design and Implementation Constraints}
There are a number of limiting factors that will constrain the Design and Implementation which are highlighted as follows:

External Storage Size: due to the system operating remotely the data store will also be remote and thus the amount of data stored will be a consideration.

RAM: the system may be operating on large amounts of data and thus must be conservative towards RAM usage at any one time.

Execution Time: the system collection, classification and visualization operations must occur within sufficient real time.

\subsection*{Dependencies}
The systems main dependency is Twitter, the system sources it's data from Twitter and thus the system, its successful delivery and continuous operation will require a reliable interface with Twitter itself.

\section*{External Interface Requirements}
\subsection*{Hardware Interfaces}
The System will have a visual website component (user interface) that will be developed to support all common device types from small constrained screen size devices such as mobile up to larger desktops.

\subsection*{Communications Interfaces}
\textbf{HTTP}: The System will require the use of HTTP connections for the consumption and exposure of data.

\textbf{SCP}: The System will require the use of the secure copy protocol for transferring the application files to the hosting destination of the remote virtual private server.

\subsection*{Interfaces}
The System will contain a command line interface and a visual interface for which the operator and general user will interact with respectively.

\subsubsection*{Command Line Interface}
The System will provide a command line interface for inputting commands within the system. Each component will have at least one runnable command e.g. for tweet collection component there will be commands for fetching historic tweets and fetching real time tweets. The main requirements of the set of these commands is that they must ensure user can view the progress of the command and that the user is presented with error messages where incorrect usage has occurred.
\\

\textbf{\underline{Progress}}

The execution time of commands may vary dependent on the command and the parameters given thus the system's operational commands must provide the user with a visual indication of progress.
\\

\textbf{\underline{Errors}}

Due to the commands being manually run the system will provide means of prompting the user with the correct parameter inputs i.e. a -h option for each command that informs the user of the required and optional parameters. The System should also inform the user of incorrect usage with commands.

\subsubsection*{Visual Website Interface}
The System's main user interface will be the visual website interface which will display all the tweets collected and the analysis conducted. 
\\

\textbf{\underline{Home View}}

This is the view in which new users will access the website from, the home page will introduce the key concepts of the site via text i.e. that the site is on the display of sentiment analysis and the site is looking at the discussion of ``Britain leaving the European Union'' via the forum of Twitter. The home page will also include the tweets for which it has most recently collected, the main navigation overlay to traverse the site and a button prompt to encourage users to read more on what the site is about.
\\

\textbf{\underline{About View}}

This is the view in which users will access when wanting to read more into how the site conducts it's analysis. The about page will clarify how the analysis was conducted from a high level and what important API's it uses in a tiled display fashion.
\\

\textbf{\underline{Analysis View}}

This is the view in which users will access when wanting to view the analysis conducted by the site, the analysis page will allow users to view analysis from the different classifications the system offers and give explanation behind each of the classifiers.
\\

\textbf{\underline{Classifier Analysis View}}

This is the view in which a user has selected a specific classifier they wish to view analysis for and are now presented with the analysis results. This view will display all the classifications for a given classifier and is viewable by time unit e.g. day/week/month/year. The view will allow for navigation by time unit using a time switcher group of buttons with a button for each of the aforementioned time units including forward and backward buttons to allow a user to go back/forth a time unit e.g. forward by one day would display the analysis of the next day after the currently selected one. Each time unit that has a lower granularity time unit represented by the system i.e. a week is composed of days, will provide the analysis of it's composite time units and provide buttons to view the analysis of each of them.

\clearpage
\section*{System Features}
Below is the catalogue of features that will be provided by the system, each feature contains it's summary description, available user interactions and it's measurable requirements.

\subsection*{1. Tweet Collection}
\subsubsection*{Feature Description}
The System must be able to collect, parse and save tweets that satisfy a selection criterion.
\subsubsection*{Stimulus/Response Sequences}
Using a command line interface; a user can give a selection criterion e.g. set of hashtags that tweets must contain and the system will collect tweets that satisfy that selection criterion.
\subsubsection*{Functional Requirements}
REQ-1.1:	The System must allow the user to specify hashtags for which to collect tweets for.

REQ-1.2: The System must collect tweets using it's given selection criterion.

REQ-1.3: The System must permanently store the tweets it collects.

\subsection*{2. Tweet Tokenization}
\subsubsection*{Feature Description}
The System must be able to tokenize tweets as part of a preprocessing step for classifications.
\subsubsection*{Stimulus/Response Sequences}
Using a command line interface; a user gives a tweet selection criteria, and the system will tokenize all tweets that satisfy the selection criteria and save the resulting tokenized tweets to the database.
\subsubsection*{Functional Requirements}
REQ-2.1: The System must provide an interface to allow users to manually tokenize tweets.

REQ-2.2: The System's tokenization interface must accept the user configurable input of a tweet selection criteria.

REQ-2.3: The System must be able to tokenize a plaintext tweet into a classifiable form.

REQ-2.4: The System must permanently store tokenized tweets that it has computed.
	
\subsection*{3. Tweet Classification}
\subsubsection*{Feature Description}
The System must be able to classify tweets against configurable classification models and store the resulting classifications for later use.
\subsubsection*{Stimulus/Response Sequences}
Using a command line interface; a user gives a selected classification model and tweet selection criteria, and the system will classify all tweets that satisfy the selection criteria and save the resulting classifications to the database.
\subsubsection*{Functional Requirements}
REQ-3.1: The System must provide an interface to allow users to manually classify tweets.

REQ-3.2: The System's classification interface must accept user configurable inputs of a tweet selection criteria and classification model.

REQ-3.3: The System must classify tweets on whether they are in favour (leave) or against (remain)  Britain Leaving the European union using brexit stance models.

REQ-3.4: The System must classify tweets on whether they are positive or negative towards Britain Leaving the European union using a sentiment polarity model.

REQ-3.5: The System must permanently store classifications that it has computed.


\subsection*{4. Visualization}
\subsubsection*{Feature Description}
The System must allow for the visualization of the tweets for which it has collected and the classifications for which it has computed.
\subsubsection*{Stimulus/Response Sequences}
Using the visualization interface a user can; 
\begin{itemize}
\item view tweets that the System has collected.
\item view the classifications that have been computed by each classification model within the System.
\item view the classifications the System has computed for a given classification model within a given time unit period.
\end{itemize}
\subsubsection*{Functional Requirements}
REQ-4.1: The System must display all Tweets that it has collected.

REQ-4.2: The System must provide a means of visually querying the resulting classifications by classification model.

REQ-4.3: The System must provide a means of visually querying the resulting classifications by time unit periods e.g. day/week/month/year.

REQ-4.4: The System must display all classifications and their associative tweets

\subsection*{5. Automation}
\subsubsection*{Feature Description}
The System must collect, tokenize, classify and visualize tweets autonomously.

\subsubsection*{Functional Requirements}
REQ-5.1: The System must conduct Tweet Collection autonomously.

REQ-5.2: The System must conduct Tweet Tokenization autonomously.

REQ-5.3: The System must conduct Tweet Classification autonomously.

REQ-5.4: The System must conduct Tweet Visualization autonomously.


\clearpage

\section*{Other Nonfunctional Requirements}
\subsection*{Performance Requirements}
NFREQ-1: The System must operate in real time and thus background operations must execute within an a max of an hour time window. 

NFREQ-2: The System must respond to HTTP Requests within 3 seconds.

\subsection*{Security Requirements}
The System will source its tweet data from the set of tweets which are publicly available or were publicly available at time of collection and thus the System abides by the twitter terms of service.

\subsection*{Software Quality Attributes}
NFREQ-3: The System must operate cross platform on the common operating systems e.g. Linux Distributions, Windows and Mac OSX.

NFREQ-4: The System visual interface must work across common browsers e.g. Safari, Chrome, Firefox and Opera.

NFREQ-5: The System must be flexible so that it operates as a framework allowing for classification models to be easily utilized in.

NFREQ-6: The System must be extensible in that the data models it stores are extendible.

NFREQ-7: The System should be reliably up and operating (and only down for when a new version is being deployed).

\end{document}