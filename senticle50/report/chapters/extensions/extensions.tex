\documentclass[11pt]{report}
\usepackage{standalone}
\graphicspath{ {images/} }
\setcounter{tocdepth}{5}
\setcounter{secnumdepth}{5}
\usepackage{natbib}
\usepackage{bibentry}
\bibliographystyle{agsm}
\usepackage{etoolbox}
\setlength{\parindent}{0em}
\setlength{\parskip}{0.25em}
\usepackage[raggedright]{titlesec}
\usepackage{hyperref}
\usepackage{capt-of}
\patchcmd{\bibliography}{\section*}{\section}{}{}
\titlespacing*{\chapter}{0pt}{-40pt}{10pt}
\titleformat{\chapter}[block]{\normalfont\huge\bfseries}{\thechapter}{15pt}{}
\usepackage{rotating}
\usepackage[titletoc]{appendix}
\usepackage{pgfgantt}
\usepackage{graphicx, rotating, caption, lscape, threeparttable}% \usepackage{amsmath}
\usepackage{array}
\usepackage{pdflscape}
\usepackage{geometry}
\usepackage{listings}
\usepackage[T1]{fontenc}
\usepackage[bottom]{footmisc}
\usepackage{multirow}
\usepackage{float}
\usepackage{subcaption}



\usepackage{etoolbox}
\usepackage{epigraph}

\setlength\epigraphwidth{11cm}
\setlength\epigraphrule{0pt}
\makeatletter
\patchcmd{\epigraph}{\@epitext{#1}}{\itshape\@epitext{#1}}{}{}
\makeatother


\begin{document}
\chapter{Conclusion}

The textual content of social media posts, in particular tweets; contain a rich source of polarised opinion that can be utilized in the Sentiment Analysis of a domain area. This dissertation has combined the method of Sentiment Analysis with tweets posted in real-time on Twitter to validate that we can indeed `implement an autonomous system to extract and visualise public sentiment from Tweets within sufficient real time'. Using the case study of the Twitter debate on ``Britain leaving the European Union'' the System has collected, analysed and visualized the sentiment of individuals within real-time. Improving on previous Systems, the utilization of Machine Learning techniques through the implementation of deep-learning models in this System has shown that deep-learning can be used in the determination of a users' sentiment within a tweet and the classification of tweets using this method can be accomplished within real-time. The enclosed work within this dissertation highlights the design, implementation and issues around the development of a real-time Sentiment extraction System and can thus be used as a reference point for future Systems of this class.
\end{document}